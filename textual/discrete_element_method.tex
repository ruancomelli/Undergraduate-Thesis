\chapter{O Método de Elementos Discretos} \label{ch:discrete_element_method}

\alert{Reescrever tudo isso}

O Método de Elementos Discretos, ou DEM\footnote{Do inglês, \textit{Discrete Element Method}}, refere-se a um conjunto de métodos numéricos aplicados na simulação de sistemas de partículas. Segundo \citeonline{bib:bicanic2007}

Cada partícula é considerada um corpo rígido com seis graus de liberdade: três translações e três rotações. O processo de simulação consiste na solução das equações diferenciais de movimento como apresentadas na \autoref{sec:equations_of_motion}. Essas equações, contudo, devem ser resolvidas para cada partícula

Os elementos do sistema, porém, interagem entre si, e as forças e os torques atuantes sobre eles dependem dessas interações.

\alert{melhorar isso}

Este capítulo visa detalhar, com base nas ferramentas e equações descritas no capítulo \ref{ch:mathematical_model}, os principais aspectos do DEM. \alert{Dar uma descrição do que é apresentado nas seções}

\section{Características Gerais do Método} \alert{Descrever aqui as características básicas e gerais que o DEM possui}

Uma distinção que pode ser feita entre métodos computacionais é a que distingue métodos explícitos de implícitos. Representando por \(y\pqty{t}\) um estado conhecido e por \(y\pqty{t+\Dt}\) um estado que se quer determinar, métodos explícitos lidam com equações da forma \alert{Usei as palavras 'distinção' e 'distingue' em sequência}
\begin{equation*}
	y\pqty{t+\Dt} = f\pqty{y\pqty{t}},
\end{equation*}
enquanto métodos implícitos buscam resolver
\begin{equation*}
	g\pqty{y\pqty{t}, y\pqty{t+\Dt}} = 0.
\end{equation*}

É evidente que os métodos implícitos, em geral, incorrem em maiores custos computacionais. 

O Método de Elementos Discretos é um método explícito. Para um \textit{passo de tempo} \(\Dt>0\), procura-se determinar o estado do sistema de partículas no instante \(t+\Dt\) a partir do estado no instante \(t\) que é, por hipótese, conhecido.

\alert{Será que essas considerações cabem aqui?}

\alert{Mostrar as equações:}

\alert{Falar algo como "de acordo com o cap. 2". Achei o segundo capítulo mal organizado, então terei que reescrevê-lo}
\alert{Introduzir algo para escrever o seguinte sistema:}
\begin{equation*}
	\left\lbrace
		\begin{array}{l}
			\accelerationi = \inverted{\massi}\cdot\dsum_{\particlej\in\neighborhoodi} \forceji\pqty{\positioni, \velocityi, \orientationi, \angularVelocityi, \positionj, \velocityj, \orientationj, \angularVelocityj}
			\\
			\accelerationi = \inverted{\massi}\cdot\dsum_{\particlej\in\neighborhoodi} \forceji\pqty{\positioni, \velocityi, \orientationi, \angularVelocityi, \positionj, \velocityj, \orientationj, \angularVelocityj}
		\end{array},\quad i = 1, \dots, \numberOfParticles
	\right.
\end{equation*}

\section{Inicialização}

\section{Condições de Contorno} \label{sec:boundary_condition}



\section{Vizinhança} \label{sec:neighborhood}

\alert{Explicar que é um tipo de método explícito no tempo, com partículas que interagem entre si}
\alert{Falar das condições de contorno: partículas com movimento pré-determinado, condição de repetição e de reflexão (parede que reflete a partícula)}
\alert{Falar das condições iniciais}
\alert{Falar do critério de parada}
\alert{Falar dos métodos de busca de vizinhos}
\alert{Mostrar os principais aspectos do algoritmo e passar por cada um deles, detalhando}
\alert{Falar da discretização do tempo}

\section{Acoplamento com outros métodos}