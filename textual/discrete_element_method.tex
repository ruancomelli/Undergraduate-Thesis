\chapter{O Método de Elementos Discretos} \label{ch:discrete_element_method}

O Método de Elementos Discretos, ou DEM\footnote{Do inglês, \textit{Discrete Element Method}}, \alert{Explicar que é um tipo de método explícito no tempo, com partículas que interagem entre si}

\section{}

Uma distinção que pode ser feita entre métodos computacionais é a que distingue métodos implícitos de explícitos. Representando por \(y\pqty{t}\) um estado conhecido e por \(y\pqty{t+\Dt}\) um estado que se quer determinar, métodos explícitos lidam com equações da forma
\[
	y\pqty{t+\Dt} = f\pqty{y\pqty{t}},
\]
enquanto métodos implícitos buscam resolver
\[
	g\pqty{y\pqty{t}, y\pqty{t+\Dt}} = 0.
\]

É evidente que os métodos implícitos, em geral, incorrem em maiores custos computacionais. 

O Método de Elementos Discretos é um método explícito. Para um \textit{passo de tempo} \(\Dt>0\), procura-se determinar o estado do sistema de partículas no instante \(t+\Dt\) a partir do estado no instante \(t\) que é, por hipótese, conhecido.

\alert{Será que essas considerações cabem aqui?}

\section{Acoplamento com outros métodos}