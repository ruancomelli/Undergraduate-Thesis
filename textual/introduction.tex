\chapter[Introdução]{Introdução}

\section{Motivação}

A compreensão acerca do comportamento de sistemas de partículas e de materiais granulados cresceu enormemente nas últimas décadas devido aos esforços da ciência e da engenharia. Tais sistemas são amplamente encontrados em processos industriais e nas ciências naturais \citeay{bib:computational_granular_dynamics}, tais como matemática aplicada, física de matéria condensada, geomecânica, agricultura, engenharia química, engenharia civil e engenharia mecânica \citeay{bib:rolling}.

Esses conhecimentos são aplicados na previsão do comportamento mecânico de geomateriais, como rochas, e da propagação de trincas e fraturas em sólidos, na simulação da produção de fármacos, alimentos, detergentes e cosméticos, no desenvolvimento de novos materiais e na modelagem da fragmentação, sedimentação, granulação, escavação, transporte e armazenamento de grãos e do escoamento em máquinas extrusoras (\citeaynp{bib:donze}; \citeaynp{bib:poschel}; \citeaynp{bib:applications}).

\alert{Imagens de aplicações aqui}

Sendo assim, surge a necessidade de se compreender a mecânica das partículas.

A análise de sistemas de partículas, contudo, costuma ser bastante complexa. Partículas podem interagir umas com as outras e com o meio em que estão imersas. As interações podem ser forças de contato, forças de corpo, trocas de calor, trocas de carga elétrica, entre outras. Ainda, a quantidade de partículas presentes no sistema de interesse pode facilmente alcançar a ordem dos milhões \citeay{bib:computational_granular_dynamics}. Além dessas dificuldades, os problemas estudados geralmente assumem geometrias distintas, tais quais a mistura de substâncias para a produção de fármacos ou o transporte de carga a granel.

Com o propósito de executar tais análises, existem três abordagens principais: a experimental, a analítica e a numérica. 

No método experimental, procura-se caracterizar os fenômenos físicos através da sua reprodução em ambientes controlados. A grande vantagem desse método é o fato de lidar com a configuração real do sistema estudado. A experimentação, no entanto, resulta em altíssimo custo e muitas vezes é limitada por questões de segurança ou pela dificuldade de reprodução do sistema real. A experimentação é utilizada para validar modelos físicos e matemáticos \citeay{bib:maliska}.

A abordagem analítica, por sua vez, busca a obtenção de soluções analíticas para o problema. A vantagem de se obterem soluções em forma fechada é seu baixíssimo custo de computação quando comparado aos outros métodos. O método analítico, porém, frequentemente depende de hipóteses simplificativas e de geometrias e condições de contorno simples, o que acaba por reduzir a aplicabilidade dessas soluções. Geralmente, utilizam-se soluções analíticas para validar soluções numéricas e auxiliar na busca de métodos mais robustos \citeay{bib:maliska}.

A fim de conciliar exatidão e eficiência, os métodos numéricos despontam como uma poderosa ferramenta capaz de resolver problemas complexos, com geometrias complicadas e grande número de partículas, alcançando, apesar disso, uma elevada rapidez e baixíssimo custo \citeay{bib:maliska}.

Sendo assim, o estudo de sistemas de partículas conta principalmente com a simulação numérica como método de análise.

\section{Revisão Bibliográfica}
 
\subsection{Sistemas de Partículas}

A definição de partícula não é um consenso entre autores, cada um adotando o conceito mais adequado à aplicação. De forma geral, compreende-se por \textit{partícula} um objeto ao qual são atribuídas propriedades físicas, mas cuja estrutura interna é desconsiderada.

\subsection{O Método dos Elementos Discretos}
\alert{Falar que DEM=Método de Elementos Discretos e que é possível acoplá-lo com outras famílias de métodos, como CFD, originando o XDEM.}

\subsection{Simuladores Existentes}

Para suprir essa necessidade, existem diversos \textit{softwares} comerciais e de código aberto, cada qual com seus objetivos, vantagens e desvantagens. Os principais simuladores existentes

\alert{Citar Rocky, LIGGGHTS e EDEM. Além disso, existem Yade, PFC (Particle Flow Code)}

\section{Objetivos e Contribuições}

\alert{Falar do interesse do SINMEC, do acoplamento, e que era necessário, primeiramente, compreender o problema e, em segundo lugar, possuir um programa que se acoplasse facilmente com trabalhos posteriores. Apresentar objetivos e objetivos específicos.}

Os objetivos deste trabalho são:
\begin{enumerate}
\item Desenvolver uma biblioteca computacional para a simulação da dinâmica de partículas através do DEM. Essa biblioteca deve possuir as seguintes características:
	\alert{Revisar essas características}
 	\begin{enumerate}
		\item Métodos para a importação e exportação de dados; 
		\item Suporte para diferentes modelos de colisão, inclusive para modelos estudados e implementados por um usuário posterior;
		\item Suporte para um número arbitrário de partículas, as quais podem assumir diferentes formas geométricas e possuir várias propriedades físicas;
        \item Suporte para a inserção de diferentes tipos de interação entre as partículas. Por exemplo, deve ser possível, com o mínimo de esforço, desenvolver simuladores que, com base na biblioteca, simulem problemas de diversas naturezas físicas, como a movimentação de planetas, a deposição de átomos sobre estruturas, a transferência de calor entre partículas, interações eletrodinâmicas, entre outras.
		\item Suporte para variadas funções de busca de partículas vizinhas;
	\end{enumerate}  
\item Implementar um simulador \alert{descrever esse simulador. Quais são seus objetivos?}
\end{enumerate}

\section{Organização do Trabalho}

\section{Notação Empregada}

Para facilitar a leitura, algumas convenções são empregadas para a representação de termos matemáticos. \alert{Quais? Falar dos vetores e das derivadas temporais}