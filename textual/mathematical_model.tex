\chapter{Modelo Matemático}

Segundo \citeonline{bib:maliska}, a solução numérica de qualquer problema físico requer sua prévia modelagem matemática. Um modelo matemático é uma representação de um sistema real através de equações. Essas equações são obtidas ao se fazerem hipóteses sobre o comportamento do sistema estudado, e a representatividade do modelo depende das simplificações feitas nesse processo.

\alert{Falar de métodos explícitos e comentar que o passo de tempo é representado por \(\Delta t\)}

\section{Equações de Movimento}

Na dinâmica de partículas, os elementos estudados são considerados corpos rígidos \citeay{bib:computational_granular_dynamics}, aos quais aplicam-se a Segunda Lei de Newton \eqref{eq:second_law_of_motion} e a equação de Euler \eqref{eq:euler_equation}.

Seja \particle{} uma partícula de massa \(m\) cuja posição é descrita, em função do tempo, pela função \(\vec{r}\).  \alert{Como abordar a questão dos ângulos de Euler?}

A Segunda Lei de Newton afirma que

\begin{equation} \label{eq:second_law_of_motion}
	\vec{F}_R = m\cdot\dv[2]{\vec{r}}{t} 
\end{equation}

\section{Modelos de Força de Colisão}
\citeonline{bib:sampaio}

\section{Extrapolação de Funções}

Nos métodos numéricos, a extrapolação de funções possui um papel fundamental por permitir a estimativa dos valores das funções além do conjunto previamente conhecido.

Para simplificação da notação, dada uma função \(y: X\to Y\), define-se
\[\drvec{n}{y} = \pqty{y, y', y'', \dots, y^{\pqty{n}}}\]

nos pontos em que todas as coordenadas estiverem definidas.

Conforme demonstrado por \citeonline{bib:extrapolation}, métodos de extrapolação lineares para uma função e suas derivadas podem ser escritos na forma

\[
\begin{pmatrix}
	\predicted{y} \\
	\predicted{\deriv{1}{y}} \\
	\predicted{\deriv{2}{y}} \\
	\vdots \\
	\predicted{\deriv{n}{y}}
\end{pmatrix}
=
\begin{pmatrix}
	a_{0,0} & a_{0,1} & \cdots & a_{0,n} \\
	a_{1,0} & a_{1,1} &  & a_{1,n} \\
    \vdots & & \ddots & \\
    a_{n,0} & a_{n,1} & & a_{n,n}
\end{pmatrix}
\cdot
\begin{pmatrix}
	y \\
	\deriv{1}{y} \\
	\deriv{2}{y} \\
	\vdots \\
	\deriv{n}{y}
\end{pmatrix}
\]

ou, de forma mais simples,

\begin{equation}
	\predicted{\drvec{n}{y}} = A \cdot \drvec{n}{y},
\end{equation}

em que a matriz \(A\) é determinada pelo método escolhido.

\alert{Falar que existem diversos métodos de extrapolação e dizer por que escolhemos o de Taylor}

\begin{theorem}[Teorema  de Taylor] \label{theo:taylor}
	Seja \(y\) uma função com derivadas \(y',\dots,y^{\pqty{k+1}}\) todas definidas em um conjunto que contenha \(\bqty{t, t+\Delta t}\), e seja \(R_{k, t, y}\) definida por
    \begin{equation*}
    	y(t + \Delta t) = y(t) + y'(t)\cdot\Delta t + \dots + \dfrac{y^{(k)}(t)}{k!}\cdot\Delta t^k + R_{k, t, y}(\Delta t).
    \end{equation*}
    Então
    \begin{equation} \label{eq:remainder_limit}
    	\lim_{\Delta t \rightarrow 0} \dfrac{R_{k, t, y}(\Delta t)}{\Delta t^k} = 0.
    \end{equation}
\end{theorem}

Uma versão mais completa desse teorema é apresentada e demonstrada por \citeonline{bib:spivak}.

A função \(R_{k, t, y}\) é o resto de ordem \(k\) para a função \(y\) no entorno de \(t\). A equação \eqref{eq:remainder_limit} indica que o resto é um termo da ordem de \(\Delta t^{k+1}\), e motiva a aproximação

\begin{equation} \label{eq:taylor_trunc}
    y(t + \Delta t) \cong y(t) + y'(t)\cdot\Delta t + \dots + \dfrac{y^{(k)}(t)}{k!}\cdot\Delta t^k.
\end{equation}

Considerando uma função \(\vec{F}:I\subseteq \real \rightarrow \real^m\), o \nameref{theo:taylor} pode ser aplicado a cada uma de suas funções coordenadas\footnote{Escrevendo \(\vec{F}(t) = \pqty{F_1(t),\dots,F_m(t)}\), a \(i\)-ésima função coordenada de \(\vec{F}\) é a função \(F_i\).}, resultando em uma expansão similar à da equação \eqref{eq:taylor_trunc}. Os casos de interesse são \(m=1\), para funções reais; \(m=2\), para vetores bidimensionais como a posição de uma partícula em uma simulação em duas dimensões; e \(m=3\), para simulações em três dimensões.

Assim, o \nameref{theo:taylor} permite a estimativa do valor de uma função em um ponto \(t+\Delta t\) a partir do valor da função e de suas derivadas em um ponto \(t\), e essa estimativa será tanto melhor quanto menor for o valor de \(\Delta t\).

Com isso, seja \(\vec{r}\) a função posição de uma partícula. Se a posição for conhecida em um instante de tempo \(t\), ela pode ser \textit{prevista} em um instante posterior \(t+\Delta t\) explicitamente:

\begin{equation} \label{eq:position_prediction}
	\predicted{\vec{r}}\pqty{t+\Delta t} = \vec{r}\pqty{t} + \Delta t\cdot\dv{\vec{r}}{t}\pqty{t} + \dfrac{\Delta t^2}{2}\cdot \dv[2]{\vec{r}}{t}\pqty{t} + \dots + \dfrac{\Delta t^k}{k!}\cdot \dv[k]{\vec{r}}{t}\pqty{t}.
\end{equation}

Não somente a posição pode ser prevista, mas suas derivadas (como a velocidade e a aceleração) também:

\begin{align}
	\predicted{\dv{\vec{r}}{t}}\pqty{t+\Delta t} & = \dv{\vec{r}}{t}\pqty{t} + \Delta t\cdot\dv[2]{\vec{r}}{t}\pqty{t} + \dots + \dfrac{\Delta t^{k-1}}{(k-1)!}\cdot \dv[k-1]{\vec{r}}{t}\pqty{t} 
    \\
	\predicted{\dv[2]{\vec{r}}{t}}\pqty{t+\Delta t} & = \dv[2]{\vec{r}}{t}\pqty{t} + \dots + \dfrac{\Delta t^{k-2}}{(k-2)!}\cdot \dv[k-2]{\vec{r}}{t}\pqty{t} 
\end{align}

Esse método de extrapolação ainda pode ser aplicado à função de orientação da partícula e a outros graus de liberdade que o problema porventura exija.

Outros métodos de extrapolação bastante utilizados são o método de Richardson e os métodos de Runge-Kutta possuem diferentes comportamentos em termos de exatidão e estabilidade \citeay{bib:gear_book}.

Com isso, é possível predizer os valores das funções de interesse de maneira bastante razoável. No entanto essa predição geralmente não é exata. Uma das razões para isto é que o truncamento da expansão de Taylor, ou qualquer outro método de extrapolação que se use, despreza a função resto, que não é necessariamente nula. Ainda assim, essa diferença é aceitável quando se utilizam passos de tempo suficientemente pequenos. 

A principal fonte de erros da equação \eqref{eq:position_prediction} é que não se considera, em nenhum momento, a ação de forças externas que porventura atuem sobre a partícula entre os instantes \(t\) e \(t + \Delta t\). É necessário, então, \textit{corrigir} a posição prevista. Essa correção pode ser feita através do algoritmo de Gear, apresentado na seção \ref{sec:gear_integration_scheme}.

\section{O Algoritmo de Gear} \label{sec:gear_integration_scheme}

\alert{Reescrever isso:}

\citeonline{bib:gear_book} considerou o problema de extrapolar funções sujeitas a equações diferenciais. Dada uma função \(y\) tal que \(y(t_0), y'(t_0),\dots, y^{(p)}(t_0)\) existem e são bem conhecidos, o objetivo é determinar os valores de \(y(t_0 + \Delta t), y'(t_0 + \Delta t),\dots y^{(p)}(t_0 + \Delta t)\) sabendo que \(y\) deve satisfazer uma equação diferencial da forma
\begin{equation*}
	y^{(p)} = f(y, y',\dots,y^{(p-1)}, t).
\end{equation*}

Do resultado desse estudo originou-se o que é conhecido por algoritmo de Gear \citeay{bib:computational_granular_dynamics}. O algoritmo consiste de duas etapas: a \textit{predição} e a \textit{correção}.

\subsection{Predição}

A etapa de predição é responsável por obter uma estimativa para \(y(t_0 + \Delta t), y'(t_0 + \Delta t),\dots, y^{(p)}(t_0 + \Delta t)\).

A predição é feita através de extrapolações, e é comum utilizarem-se extrapolações polinomiais. Dentre os principais métodos de extrapolação polinomial estão a extrapolação por expansão de Taylor, extrapolação de Richardson e interpolação de Aitken \citeay{bib:gear_book}.

\subsection{Correção}

Primeiramente, define-se uma função de erro

\begin{equation*}
  F\pqty{\vec{a}} = \dfrac{\Delta t^p}{p!}\cdot f(a_0, \dfrac{a_1}{\Dt}, \dfrac{2a_2}{\Dt^2},\dots,\dfrac{\pqty{p-1}!\cdot a_{p-1}}{\Dt^{p-1}}, t) - a_p
\end{equation*}

\alert{Continuar}
 