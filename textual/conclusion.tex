\chapter{Conclusão} \label{ch:conclusion}

\section{Sumário}

Neste trabalho, foram apresentados os fundamentos do Método de Elementos Discretos, a implementação de uma biblioteca de códigos e resultados de simulação.

No \cref{ch:mathematical_model}, foram apresentados os aspectos matemáticos, como as equações de movimento baseadas nas leis de Euler para o movimento, os modelos de força de colisão e o algoritmo de integração de Gear. Dentre os modelos de força, foram detalhadas as expressões de cálculo e o significado físico dos parâmetros para os modelos de amortecimento linear e de esferas viscoelásticas, para a força normal, e os modelos de Haff e Werner e de Cundall e Strack, para a força tangencial. Também foi apresentado o algoritmo de Gear, responsável pela solução das equações diferenciais em cada passo de tempo.

Por sua vez, no \cref{ch:discrete_element_method}, foram apresentados os métodos de elementos discretos em maiores detalhes. Foram apresentados a discretização temporal e os principais elementos pertencentes às simulações. Cada etapa do algoritmo foi explicada, demonstrando, com base em fluxogramas e listagens, o seu funcionamento. O primeiro passo do algoritmo é a inicialização, em que a configuração da simulação é obtida de arquivos de entrada. Depois, inicia-se o laço temporal em que são resolvidas, para cada instante do tempo discretizado, as equações de movimento. A solução das equações de movimento foi descrita e uma listagem, apresentada. A exportação de dados, também, é um passo essencial na simulação visto que permite, de fato, a obtenção dos resultados simulados. O laço de simulação é executado repetidamente até que um critério de parada previamente estabelecido seja satisfeito, levando o programa à etapa de finalização. 

Ainda, foi apresentada a importância dos métodos de monitoramento de vizinhanças para simulações com elevado número de partículas. Também foi discutida a limitação do passo de tempo que, embora necessária, não segue, necessariamente, um método, dependendo mais da experiência do usuário que de critérios matemáticos rigorosos. Além disso, foram indicadas as condições de contorno e de restrição mais usuais, como a condição de movimento prescrito aos elementos de contorno, de paredes reflexivas para as bordas do domínio e de restrição de movimento relativo entre subelementos de partículas compostas. Mais ainda, no \cref{ch:discrete_element_method} foram apresentados aspectos mais específicos concernentes à simulação da fragmentação de partículas e à paralelização dos métodos de elementos discretos. Por fim, foi apresentado brevemente o acoplamento que pode ser feito entre o \DEM{} e outras famílias de métodos computacionais, como o Método de Elementos Finitos e a Mecânica de Fluidos Computacional.

A implementação da biblioteca e do simulador utilizados na obtenção de resultados foi então apresentada no \cref{ch:results}. A partir disso, foram considerados quatro problemas: o problema do lançamento oblíquo, o da esfera quicando, o da colisão entre esferas e, por fim, o da queda com arrasto.

\section{Conclusões}

A análise dos resultados obtidos no \cref{ch:results} indica uma grande concordância entre as simulações e as soluções analíticas dos problemas considerados.

O primeiro problema tratou do lançamento oblíquo de uma partícula. Embora relativamente simples e de fácil resolução analítica, esse problema permite a validação do simulador em problemas em que a força atuante sobre as partículas é constante. De fato, o erro apresentado pelos resultados numéricos é da ordem do erro de máquina, o que é a máxima exatidão que se pode exigir de um algoritmo computacional.

O segundo problema, ligeiramente mais complexo, inclui, além da gravidade, um elemento plano correspondente ao chão com o qual a partícula colide. Com esse problema, foi possível a análise da energia da partícula e do coeficiente de restituição resultante das colisões em um caso conservativo, em que não há componentes dissipativas na expressão para a força, e um caso dissipativo, em que essas componentes são incluídas. Em ambas as situações, os resultados numéricos reproduziram o comportamento analítico do sistema. Observou-se, porém, uma queda de energia no caso conservativo, o que não existe na solução analítica. Entretanto esse erro e o erro do coeficiente de restituição são, cada um, quatro ordens de magnitude inferiores aos valores esperados, o que pode ser explicado pela inexatidão das aproximações consideradas no método numérico empregado. Também é interessante notar a discrepância entre os coeficientes de restituição analítico e numérico no caso dissipativo quando a partícula se aproxima do repouso. Isso se deve ao fato de que a velocidade relativa entre os elementos colidentes tende a zero, e a computação do coeficiente de restituição deixa de fazer sentido. Esse repouso é atingido em decorrência da dissipação de energia mecânica nos choques como resultado da introdução das constantes de amortecimento no cálculo das forças.

No terceiro problema foi abordada a colisão entre duas partículas em três casos: um conservativo, um dissipativo sem rotações e outro dissipativo com rotações. O caso conservativo demonstrou um excelente desempenho, em que tanto a quantidade de movimento linear quanto a energia se conservaram, como esperado. O erro máximo do coeficiente de restituição atingiu, nessa situação, uma fração de bilionésimos da solução analítica. É interessante, ainda, observar-se que a força de contato entre as esferas ultrapassou \SI{100}{\kilo\newton} em módulo como consequência da rigidez das partículas, do modelo de interação usado e da configuração do problema. Também ficou evidente a transformação de energia cinética em energia elástica e vice-versa durante o choque.

No caso dissipativo sem rotação do terceiro problema, observou-se, novamente a conservação da quantidade de movimento linear visto que esta independe do modelo de força usado, mas segue da Lei de Ação e Reação. A energia cinética, por sua vez, reduziu-se como resultado da dissipação na colisão. Novamente, a força máxima no contato ultrapassou os \SI{100}{\kilo\newton} em módulo, mas, dessa vez, apresentou um perfil diferente, assimétrico, resultante da dissipação de energia. Ainda assim, o coeficiente de restituição numérico aproximou-se bastante do analítico, evidenciando a corretude do algoritmo. É interessante ainda observar o caráter crescente do erro máximo do coeficiente de restituição em função do passo de tempo empregado. Esse erro vai da ordem de milionésimos para simulações com \(\Dt=10^{-8}\si{\second}\) para mais que \SI{100}{\percent} quando \(\Dt=10^{-5}\si{\second}\), caso em que o passo de tempo se aproxima do tempo de colisão.

No caso dissipativo com rotação, observou-se, mais uma vez, o comportamento esperado para as grandezas translacionais: a quantidade de movimento linear se conservou e houve dissipação de energia, enquanto o coeficiente de restituição manteve-se próximo do analítico. Além disso, notou-se, nessa situação, a conservação da quantidade de movimento angular e uma queda na energia rotacional. Por não haver componente elástica no modelo de força tangencial, não houve recuperação de energia rotacional ao fim da colisão. Percebe-se, ainda, que a força tangencial consequente da rotação dá início a um movimento vertical, sendo que, também nessa direção, a quantidade de movimento linear se conserva. Mais uma vez, a força de contato horizontal atinge a casa dos \SI{100}{\kilo\newton}, ao passo que a força de contato vertical ultrapassa os \SI{20}{\kilo\newton}.

Por fim, foi apresentado o problema da queda com arrasto, em que uma partícula foi liberada no espaço a \SI{3}{\kilo\meter} de altura sob ação da força da gravidade e da resistência do ar. Da comparação do histórico da velocidade da partícula com a velocidade terminal analiticamente calculada, infere-se a corretude do algoritmo mesmo em situações em que a natureza da força é bem distinta das forças de colisão e da força constante da gravidade. A energia cinética da partícula cresce até se estabilizar em, aproximadamente, \SI{1}{\kilo\joule} aos \SI{30}{\second} de simulação, quando as forças de arrasto praticamente se igualam à força gravitacional. A energia mecânica, por outro lado, é continuamente dissipada em virtude da resistência do ar.

A importância desse último problema reside no fato de que o mesmo fenômeno físico é considerado na simulação acoplada entre o \DEM{} e a mecânica de fluidos, a diferença consistindo apenas na expressão para o cálculo do coeficiente de arrasto que, no caso geral, depende do número de Reynolds, enquanto, neste trabalho, foi considerado constante.

\section{Recomendações para Trabalhos Futuros}

Para trabalhos futuros nessa linha de pesquisa, sugere-se:
\begin{alineas}
\item remover as restrições indicadas na \cref{sec:computational_implementation} da biblioteca. Essa tarefa pode ainda ser dividida em etapas:
	\begin{alineas}
	\item permitir o uso de algoritmos de integração distintos, e mesmo da escolha de diferentes métodos de extrapolação no algoritmo de Gear;
	\item possibilitar a escolha de critérios de parada arbitrários;
	\item acrescentar suporte a métodos de monitoramento de vizinha quaisquer;
	\item generalizar o procedimento de entrada e saída de dados para permitir formatos de arquivo diferentes;
	\item permitir que não apenas a posição e a orientação sejam submetidas à integração, mas também graus de liberdade quaisquer.
	\end{alineas}
\item melhorar o pós-processamento. Embora a linguagem \Python{} forneça diversas ferramentas apropriadas para o processamento dos resultados das simulações, existem bibliotecas e \textit{softwares} especializados que o fazem de maneira muito mais eficiente. Por exemplo, sugere-se, aqui, o uso do Paraview \cite{bib:paraview_book,bib:paraview_handbook}, que possui ferramentas de visualização bastante completas e suporte para scripts em \Python{};
\item embora a classe template \lstinline[style=Inline C++]{Simulator} definida na biblioteca aceite tipos de entidades e de interações quaisquer, ainda pode ser trabalhoso de se implementarem as classes de interesse a serem inseridas no simulador. Propõe-se a implementação de tipos de partículas adicionais, como os superelipsoides, descritos em \citeonline{bib:sampaio}, as partículas poliédricas e as compostas, descritas por \citeonline{bib:computational_granular_dynamics};
\item introduzir a definição de materiais. Até o momento, todas as propriedades físicas de interesse devem ser informadas manualmente nos arquivos de entrada. Entretanto esses valores podem ser obtidos se se souberem os materiais de que são compostos os elementos da simulação. Para tanto, o conceito de \textit{materiais} deve ser introduzido na biblioteca, e um banco de dados deve ser inserido;
\item adicionar unidades de medida às propriedades. Na implementação corrente, o programa opera as variáveis independentemente de suas unidades de medida as quais devem ser rastreadas manualmente pelo usuário;
\item estudar o acoplamento entre \DEM{} e \CFD{} como indicado na \cref{sec:discrete_element_method:coupling_with_other_methods}. Propõe-se, ainda, que se considere o uso do Método de Volumes Finitos Baseado em Elementos para a resolução do escoamento;
\item incluir modelos de troca de calor como aquele descrito por \citeonline{bib:simsek2009}.
\end{alineas}

